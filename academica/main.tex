% the option beamer without any argument specifies the beamber class, otherwise by default is used book
\documentclass[]{academica}

\title{Title \\ example}
\author{Me}
%\course{master degree in Artificial Intelligence and Cybersecurity}
\supervisor{Prof.\ }
\cosupervisor{Prof.\ \and Prof.\ }
\tutor{Prof.\ }
\date{}
% \rights{}
\contacts{}
% \contacts{\textsc{Institute Contacts}\\
% Universit\'a degli Studi di Udine - Dipartimento di Scienze Matematiche, Informatiche e Fisiche\\
% Via delle Scienze, 206, 33100 Udine (IT)\\
% +39 0432 558400\\
% \url{https://www.dmif.uniud.it/}}

% \usepackage[a-1b]{pdfx}
\usepackage[pdfa]{hyperref}

%% --- Stili di pagina disponibili (comando \pagestyle) ---
%% sfbig (predefinito): Apertura delle parti e dei capitoli col numero grande; titoli delle parti e dei capitoli e intestazioni di pagina in sans serif.
%% big: Come "sfbig", solo serif.
%% plain: Apertura delle parti e dei capitoli tradizionali di LaTeX; intestazioni di pagina come "big".

\begin{document}

\maketitle

\frontmatter

\begin{dedication}
	
\end{dedication}

\acknowledgements

\abstract

\tableofcontents

\mainmatter

\chapter{First chap}

\section{First sec}

\subsection{Subsection}

\subsubsection{SubSubsection}

\appendix

\chapter{First chap of appendix}

\backmatter

%%\summary
%
%%% Bibliografia (praticamente obbligatoria)
%\bibliographystyle{plain_\languagename}%% Carica l'omonimo file .bst, dove \languagename è la lingua attiva.
%%% Nel caso in cui si usi un file .bib (consigliato)
%\bibliography{academica}
%%% Nel caso di bibliografia manuale, usare l'environment thebibliography.
%
%%% Per l'indice analitico, usare il pacchetto makeidx (o analogo).

\end{document}
