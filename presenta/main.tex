% !TeX program = lualatex

% CLASS OPTIONS (all optionals, order is irrelevant)
%   - pdfa: specifies if the pdf should be of `pdfa` type
%   - banner={banner1,banner2,...}: specifies the banners to use (order of the banner is important)
%   - handout: removes the \pause
%   - nolight: specifies if you dont want a light font (default)
\documentclass[banner={uniud},handout]{presenta}

\title{Some simple title}
\author{me}

\subtitle{The subtitle}
\supervisor{Prof A}
\cosupervisor{Prof B, Prof C}
\tutor{Prof D}

\begin{document}

\maketitle

\tableofcontents

\section{Section}

\begin{frame}{A long frame title with subtitle}{a longer subtitle}
  Here's a list of stiles:
  \begin{itemize}
    \item regular;
    \item \textsl{slanted};
    \item \textit{italic};
    \item \emph{emphasized};
    \item \textsb{semibold};
    \item \textsi{semibold-italic};
    \item \textbf{bold};
    \item \textbi{bold-italic}.
  \end{itemize}
\end{frame}

\begin{frame}[right]{Short}{srt}
  A \textit{right sided} frame title with \textbf{enumerate}:
  \begin{enumerate}
    \item item 1;
    \pause
    \item second item;
    \pause
    \item last item.
  \end{enumerate}
\end{frame}

\begin{frame}{Figure example}
  Here we import a figure:
  \begin{figure}[H]
    \input{tikz/example.tikz}
    \caption{An example of \texttt{tikz}.}
  \end{figure}
\end{frame}

\begin{frame}
  \begin{columns}
    \hfill
    \begin{column}[t]{0.40\paperwidth}
      \justifying
      An example of frame without title, with a split layout and a citation~\cite{example1}.
    \end{column}
    \hfill
    \begin{column}[t]{0.40\paperwidth}
      \justifying
      All this text is justified\footnotemark, and height is adjusted on higher text level.
    \end{column}
    \hfill
  \end{columns}
  \footnotetext{simple footnote}
\end{frame}

\section{Another helpful section}

\begin{frame}{Math examples}
  This is an addition of values $0x3 + 0^3 + I_{m^\prime} - \rho^{24} = \widehat{16}$, followed by this equation:

  \begin{equation*}
    f = - 5x^3 + 7x^2z^2 + 4xy^2z + 4z^2  
  \end{equation*}

  that results in this $x^2 \notin \langle \mathsf{LT}(f_1), \mathsf{LT}(f_2), \ldots \rangle$.

  \begin{definition}[My important definition]
    The function $S: \mathbb{F}^2_p \rightarrow \mathbb{F}^2_p$ is a common function that loads a constant $\gamma$ and is defined as follows:
    \begin{equation*}
      S_i(x_i) \rightarrow \left( x_{i-1} + \frac{(x_{i-2})^2}{\sigma} + \gamma_i \right),
    \end{equation*}
    with $\sigma$ another constant.
  \end{definition}
\end{frame}

\standout{If you wanna make text standout :)}

\references

\end{document}
